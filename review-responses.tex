%%%%%%
%%
%%  Don't reorder the reviewer points; that'll mess up the automatic referencing!
%%
%%%%%

\begin{minipage}[b]{2.5in}
  Resubmission Cover Letter \\
  {\it Theoretical Population Biology}
\end{minipage}
\hfill
\begin{minipage}[b]{2.5in}
    Todd Parsons \\
    \emph{and} Peter Ralph \\
  \today
\end{minipage}
 
\vskip 2em
 
\noindent
{\bf To the Editor(s) -- }
 
\vskip 1em

We are writing to submit a revised version of our manuscript,
``Large effects and the infinitesimal model''.
We have addressed the remaining points of the reviwers,
hopefully to everyone's satisfaction.

Along with our revised paper and the responses to reviewers,
we are also submitting a PDF that has both changes to the paper since the first submission highlighted,
and hyperlinked reviewer comments.
(Line numbers in the standalone responses to reviewers
refer to the PDF of the paper without changes highlighted,
while line numbers in the merged document with changes highlighted are self-consistent.)


\vskip 2em

\noindent \hspace{4em}
\begin{minipage}{3in}
\noindent
{\bf Sincerely,}

\vskip 2em

{\bf 
Todd Parsons and
Peter Ralph
}\\
\end{minipage}

\vskip 4em

\pagebreak

%%%%%%%%%%%%%%
\section*{AE comments:}

\begin{quote}
It would be good to try to address the remaining concern of Reviewer 1 and to
clarify the corresponding assumption on which your analysis seems to rest.
After that, the paper will be accepted for publication without further
reviewing.  
\end{quote}

Thanks very much, and apologies such a minor change took some time.
We've tried our best here.


%%%%%%%%%%%%%%
\reviewersection{1}

\begin{point}{}
I don't think the authors quite address the point I had raised, and their
response and also the change in the paper that they make \revref{} is
rather cryptic and hard to understand. I think all that I was trying to say was
that the independence of allelic states, i.e., the LE assumption, is a big part
of their argument but not of the original infinitesimal model or of the TPB
2017 paper. And I thus felt that it is important to highlight the rather
limited nature of their proposition, which I don't think quite comes across.
\end{point}

\paragraph{Reply:}
Although what you're saying here seems very clear,
we must be talking past each other somehow,
since \citet{barton2017infinitesimal} unambiguously \emph{does} assume LE,
for at least the main results on
``the infinitesimal model as the limit of Mendelian inheritance''.
In \citet{barton2017infinitesimal}, Assumption (4) (``Ancestral population'')
on page 58 of the TPB PDF,
a prerequisite to the main convergence results, says:
``\emph{Although it is not strictly necessary,
we assume that in generation zero, the individuals that
found the pedigree are unrelated. They are sampled from
an ancestral population in which all loci are assumed to be
in linkage equilibrium.}''

Of course, loci in \emph{subsequent} generations won't be in LE,
and we assumed in our previous revision that this is what the reviewer was referring to.
Our argument here is that if the infinitesimal model
is not a good model for this simplest LE situation,
then it won't be in subsequent generations also.

Perhaps by ``the original infinitesimal model or the TPB 2017 paper'',
the reviewer means something different than these convergence results?
We certainly agree that \emph{the infinitesimal model},
stated as a phenomenological model for trait inheritance,
does not make any assumptions whatsoever about the precise genetic basis
(including linkage equilibrium or not).
However, that's probably not what the reviewer means, since that's entirely
beside the point (as in this section of the paper we're looking at
consequences of the genetic basis)?

In summary, we're probably missing something,
but despite significant thinking, haven't figured out what it might be.
We've tried (again) to clarify the general point;
maybe it will help.

