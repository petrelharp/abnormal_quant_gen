%%%%%%
%%
%%  Don't reorder the reviewer points; that'll mess up the automatic referencing!
%%
%%%%%

\begin{minipage}[b]{2.5in}
  Resubmission Cover Letter \\
  {\it Theoretical Population Biology}
\end{minipage}
\hfill
\begin{minipage}[b]{2.5in}
    Todd Parsons \\
    \emph{and} Peter Ralph \\
  \today
\end{minipage}
 
\vskip 2em
 
\noindent
{\bf To the Editor(s) -- }
 
\vskip 1em

We are writing to submit a revised version of our manuscript,
``Large effects and the infinitesimal model''.
We have addressed the remaining points of the reviwers,
hopefully to everyone's satisfaction.

Along with our revised paper and the responses to reviewers,
we are also submitting a PDF that has both changes to the paper since the first submission highlighted,
and hyperlinked reviewer comments.
(Line numbers in the standalone responses to reviewers
refer to the PDF of the paper without changes highlighted,
while line numbers in the merged document with changes highlighted are self-consistent.)


\vskip 2em

\noindent \hspace{4em}
\begin{minipage}{3in}
\noindent
{\bf Sincerely,}

\vskip 2em

{\bf 
Todd Parsons and
Peter Ralph
}\\
\end{minipage}

\vskip 4em

\pagebreak

%%%%%%%%%%%%%%
\section*{AE comments:}

\begin{quote}
As you will read from the reviewers' comments below, both of them find that the quality and clarity of the exposition has improved a lot, and they only have a list of minor (nonetheless important) comments that should be addressed in a minor revision. Provided the points raised are suitably addressed, the paper will then be accepted for publication. And though this has nothing to do with the paper, happy New Year to the authors ;-).
\end{quote}

Happy New Year!


%%%%%%%%%%%%%%
\reviewersection{1}


\begin{point}{}
     As far as I understand, the crucial assumption is that the effect sizes $\eta_{l}^{i}$
     of the alleles carried by an individual $i$ are independent
     (as in the first line of Proposition \ref{prop:parental_contribs}),
     which basically amounts to assuming linkage equilibrium in the population.
     In that case, of course, all trait values across any group of individuals are multi-variate Gaussian
     (provided effect size distributions are not heavy-tailed),
     without needing to condition on any ancestors.
     Moreover, under these conditions,
     even the trait value distribution in the population would be Gaussian
     (provided effect size distributions is not heavy-tailed).
     As far as I understand, \citet{barton2017infinitesimal} is more general
     in that it does not require $\eta_{l}^{i}$ to be independent,
     except in the founder $t=0$ generation
     (and I am not sure how crucial even that assumption is);
     in later generations, there can be an arbitrary amount of LD.
     This is a crucial point, since in general, under the infinitesimal model,
     non-Gaussian population distributions only arise because of LD, in which case,
     the $\eta_{l}^{i}$ are not independent.
     Thus, it is worth highlighting that assuming $\eta_{l}^{i}$ to be independent
     basically implies linkage equilibrium, and that this is in general,
     not a requirement for the infinitesimal model
     or for the derivations of \citet{barton2017infinitesimal},
     whereas it is a key requirement here.
     I think this had been a big part of my earlier confusion.
\end{point}

\reply{
    It is true that if the $\eta$ are not independent then $Z_M$ and $V_M$ are not independent,
    and so the Kac-Bernstein theorem does not apply.
    However -- returning to the goal of this section here -- it seems hard and maybe impossible
    to use LD to obtain a inifinitesimal-like, non-Gaussian model.
    But, we also don't see why it's impossible, and have added a note to this effect \revref.
    (Also we'd like to point out that the independence only applies to the \emph{parental} generation;
    there is in fact LD in the offspring's generation, which is the generation Proposition \ref{prop:parental_contribs}
    refers to. Our results could be extended to further generations, as we remark just before \revref,
    but since our point is about non-existence of other models, this seemed superfluous.
    Note also that although we assume independence, we don't assume the $\eta$ are identically distributed,
    so that a Gaussian or other stable distribution is not guaranteed.)
}

\begin{point}{}
     I think this also needs to be emphasised in later sections,
     e.g., section 4, where the statement that no assumption is made about the initial distribution of the population
     doesn't seem to be quite right.
     I think the implicit assumption of Linkage equilibrium does come in
     once one assumes that all the $\eta_{l}^{i}$ are independent.
\end{point}

\reply{
    We don't think this applies, as
    Section 4 is explicitly not working with the model of Section 3 --
    there are no loci explicitly modeled at all in Section 4.
    We have added a reminder to emphasize this fact \revref.
}

%%%%%%%%%%%%%%
\reviewersection{2}


\begin{point}{\revref}
     p.4, line 3 (from top): garbled sentence
\end{point}

\reply{
    We've simplified the sentence.
}

\begin{point}{\revref}
     delete one `that'
\end{point}

\reply{
    Done.
}

\begin{point}{}
     Section 6: it is unfortunate to use $\mu$ for the mutation probability and $\mu(x)$ for the death rate. Maybe use $u$ for the mutation prob.
\end{point}

\reply{
    Good call; done.
}

\begin{point}{\revref}
     p.18, line 460: delete one comma
\end{point}

\reply{
    Whoops! Done.
}
