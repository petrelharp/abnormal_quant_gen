\documentclass[11pt]{article}

\usepackage{fullpage}

\begin{document}

\section{Editor Comments}

The two reviewers find the results interesting and worth being eventually published in the special issue. However, they highlight a number of places in the text where the properties claimed or the precise questions addressed need serious clarification. As the readership of the special issue will be quite broad, it is important that the main messages are accessible to colleagues who do not necessarily know Fisher's infinitesimal model precisely. Therefore, although the mathematical results in themselves do not require deep modifications, I recommend that a major revision of the paper should be provided, which would take into account the different comments and suggestions of the reviewers.

\section{Reviewer Comments}

\subsection{Reviewer \#1:} 

The paper explores inheritance of an additive trait influenced by a large number of loci with effect sizes drawn from a power law distribution, also fitting effect size distributions as estimated by GWAS (for several human disease traits) to estimate plausible power law exponents. The study is motivated by Barton et al's 2017 paper which shows how in the limit of a very large number of loci with effect sizes drawn from a distribution with finite variance, inheritance is described by the the infinitesimal model: in particular, that the distribution of trait values of the offspring of two parents is multivariate normal with a variance-covariance matrix that is independent of parental trait values. This paper explores whether some analogous statements can be made if the underlying effect size distribution does not have finite variance- whether, loosely speaking, some ``generalisation'' of the infinitesimal model might still apply. Below I summarise some of my concerns about the paper.

\begin{itemize}
\item Page 2, paragraph starting ``This suggests exploring whether stable distributions..''
The questions as stated: e.g., whether stable distributions can ``stand in'' for a Gaussian distribution in the infinitesimal model or whether the results of Barton et al ``carry over'' when the sum of effect sizes follows a stable law are quite vague and allow for multiple interpretations. Ideally, in order to address whether Barton et al's results (do not) carry over, one would like to at least show that with a power law distribution of effect sizes, the ``Mendelian noise term'', i.e., the difference between the offspring trait value and the midparent value, is (not) independent of the midparent value in the limit of a very large number of loci. And indeed this is what the authors address
somewhat non-rigorously (and for the case of a Cauchy distribution of effect sizes) in section 5 and using simulations : see e.g., figures 4C and 4f, figures 6C and 6F. Unsurprisingly, these simulations show that with e.g., Cauchy distribution of effect sizes, the non-independence between the segregation/noise term and the midparent trait value does not hold. However, a lot of the mathematical analysis in the paper (sections 3 and 4) explores when the distribution of trait values in a population is (not) Gaussian, which is a completely different question.

In the same paragraph: I don't think the claim that ``independence of offsprings' deviations and midparent values implies a Gaussian distribution'' (which is also stated in the abstract) is true? Surely, ``independence of offsprings' deviations and midparent values'' is just the basic statement of the infinitesimal model (say, in Barton et al 2017) and the infinitesimal model does not imply a Gaussian distribution of trait values? See also section 3.

\item Section 2 and figure 1: It is not obvious to me why one would study the ``frequency-weighted'' effect sizes. There is an assumption here I think that allele frequencies are uncorrelated with effect sizes: this will not be true if for example traits are under some kind of stabilising selection: see e.g., Simons et al (A Population Genetic Interpretation of GWAS Findings for Human Quantitative Traits) for a detailed exploration of this. Moreover, if SNPs are not truly causal but only tag causal alleles, then weighting by allele frequencies can be even more problematic?

As an alternative, would it make sense to look at what fraction of loci have effect size difference between alternative alleles to be $>2t$ without weighing by allele frequency?
More broadly, how do we interpret the frequency-weighted vs. unweighted distributions?

I was also wondering about the ``goodness of fit'' of observed distributions to power laws. I am not necessarily recommending a detailed quantitative analysis, but nevertheless it may be useful to comment on whether one has power to say reject alternative distributions (e.g., a mixture of exponentials) that are often considered. If not, then perhaps the exact exponents inferred are unimportant, and what the data points to more broadly, is a broad distribution of effect sizes.

\item Section 3, Sentence starting ``We know from Barton et al (2017)...'' : As far as I understand, the statement of Barton et al is that the joint distribution of trait values of offspring is multivaraite normal, conditioned on the two parents. I don't think the joint distribution of two randomly picked parents and their offspring is mutivariate normal, as this would require some kind of normality assumption about trait values in the population as a whole. It'd be helpful to distinguish more carefully between conditioned and unconditioned distributions, especially in this section.

In general, I find this section rather confused: the infinitesimal model basically implies that the ``Mendelian segregation term'' $R_{M}$ (which is the difference between an individual and the midparent value $\bar{Z}_{M}$) is normally distributed with mean zero and variance that depends on the relatedness between parents but is independent of the midparent value $\bar{Z}_{M}$. However, in proposition 1 (page 7), the authors claim that $\bar{Z}_{M}$ and $R_{M}$ are jointly Gaussian (which is equivalent to saying that the trait value of an individual and the mean of its parents' trait values is jointly Gaussian), which is in general not true.

I think it is also not correct to claim that ``Since models can be easily set up for which trait distributions are not Gaussian, the implication of this is that for such models, independence of the Mendelian sampling term is not likely a good assumption.'' (just below Proposition 1). As far as I understand, a non-Gaussian distribution of trait values can arise quite easily under the infinitesimal model (where the Mendelian segregation term is independent of the midparent value) e.g, due to strong or non-Gaussian selection or migration or some combination of the two (see, e.g., Figure 1 in Barton et al).

\item Section 4, first paragraph: I don't think the third assumption (that trait values in the population follow a Gaussian distribution) is a component of the ``infinitesimal model''. I think Turelli (2017) is also quite clear about this stating that a Gaussian distribution of trait values should only emerge for Gaussian or very weak selection on the population.

More generally, I found it a bit hard to follow the overall logical flow of section 4.1, especially once the authors launch into an exploration of the ``reproduction'' and ``noise'' terms (pages 9 and 10).
It would be useful to clarify what the biological/intuitive meaning of the noise and reproduction terms at the outset. It would also be useful to state (at the beginning) what the main goal of these explorations is: (is the goal to identify when both terms ``have well-posed limits independent of the other'' as stated at the end of page 8), and also to summarise (at the end) to what extent one can do this in all generality.


\item Last inequality on Page 13: Is this the distribution of the largest of the $M$ alleles carried by an individual, conditioned on the trait value $Z$ of the individual (and this is independent of $Z$?), or is it just the largest of $M$ iid draws from a Cauchy distribution? If the former, then it'd be good to say this explicitly. If the latter, then this does not quite answer the question posed at the beginning of the section about how much information knowing the trait value of an individual gives about underlying allelic effects.

Also, maybe worth specifying what the corresponding distribution for the largest of $M$ alleles looks like when the effect size distribution has finite variance, in order to highlight the contrast between ``well-behaved'' and heavy-tailed effect size distributions.
\end{itemize}

\subsubsection{Minor comments:}

\begin{itemize}
\item Introduction: Also worth citing Fisher, Bulmer etc. (the original references) when introducing the infinitesimal model?

\item Page 2, line 4: ``Perhaps the distribution of effect sizes within each gene is Normal'': I am confused by this. Do you mean that the distribution of effect sizes across all genes (with roughly the same ``proximity'' to the trait) is normal. If yes, then the above phrasing is a bit misleading. If what is meant that is really that the effect sizes of different alleles within a gene are normally distributed, then I am not sure if this is entirely plausible: see various papers by Turelli (e.g., ``Heritable Genetic Variation via Mutation-Selection Balance: Lerch's Zeta Meets the Abdominal Bristle.''). Presumably, a normal distribution of effect sizes for a sufficiently large genomic region is a good approximation but it is unclear what is large enough: smaller or larger than a typical gene...?

\item Page 2, sentence starting ``In fact, many traits claimed...'': This phrasing here is rather cryptic and I do not understand what is being said. Maybe rephrase? Also a reference would help...

\item Figure 2: Is there also a correlation between the number of cases and the number of SNPs? In other words, to what extent are (b) and (c) independent?

\item Page 12, sentence starting ``To see what knowing the parental trait value...'': In what sense is this the ``parental'' trait value??

\item Last inequality on Page 13: should the subscript be $j$ instead of $i$?

\item Simulations: Why not choose the same effect size distributions in the neutral case and the case with stabilizing selection? Alternatively, if different effect size distributions are used, it may be good to plot the distribution of (appropriately) scaled trait values. Right now the figures give the impression that more variation is maintained under stabilizing selection..

\item Figure 4 caption: I find the phrasing ``(a,b) chosen to span the 5\% of midparents centered on the 10\% quantile of midparent value'' a bit confusing. Does this mean a and b are respectively the 7.5\% and 12.5\% quantile of the distribution of midparent trait values?

\item Discussion: In the context of theory combining large and small-effect loci, the 2008 paper by Chevin and Hospital: ``Selective Sweep at a Quantitative Trait Locus in the Presence of Background Genetic Variation'' and the 1983 work by Lande: ``The response to selection on major and minor mutations affecting a metrical trait'' may also be somewhat relevant.
\end{itemize}

\end{document}